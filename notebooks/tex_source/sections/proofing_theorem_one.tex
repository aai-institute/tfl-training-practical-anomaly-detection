One may wonder how the statement of the Fisher-Gnedenko-Tripet theorem is
obtained without providing bounds on convergence. The reason is that the
limiting distribution of (renormalized) maxima must have a very special property
\textemdash\ it must be max-stable. It is instructive to go through a part of the proof to
get a feeling for the EVT theorems. We will do so in this exercise.


\begin{definition} 
    A cumulative distribution function $D(z)$ is called \textit{max-stable}
    iff for all $n\in\mathbb{N} \ \exists \ \alpha_n>0, \beta_n \in  \mathbb{R}$
    such that 

    \begin{equation}
        D^n(z) = D(\alpha_n z + \beta_n).    
    \end{equation}

\end{definition}

\medskip

Prove that from $\lim_{n\rightarrow \infty} P\left( \frac{M_n - b_n}{a_n} < z
\right) = G(z)$ follows that $G(z)$ is max-stable.

\medskip

This goes a long way towards proving the first EVT theorem. One can easily
compute that the GEV distribution is max-stable and with more effort one can
also prove that any max-stable distribution belongs to the GEV family. Thus, the
proof of the theorem is very implicit and does not involve any convergence rates
or bounds.
